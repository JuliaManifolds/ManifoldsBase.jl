
% JuliaCon proceedings template
\documentclass{juliacon}
\setcounter{page}{1}
\usepackage{amsmath,booktabs,mathtools,amssymb}
\newcommand{\term}[1]{\emph{#1}}
\newcommand{\ManifoldsBaseVersion}{2.0}

\hypersetup{colorlinks=true}
%
\begin{document}

% **************GENERATED FILE, DO NOT EDIT**************

\title{ManifoldsBase.jl – an interface for manifolds in Julia}

\author[2]{Mateusz Baran}
\author[1, 2]{Ronny Bergmann}
\affil[1]{Department of Mathematical Sciences, Norwegian University of Science and Technology, Trondheim, Norway}
\affil[2]{AGH University of Science and Technology, Kraków, Poland}

\keywords{Julia, Riemannian manifolds, Lie groups, differential geometry, numerical analysis, scientific computing}

\hypersetup{
pdftitle = {ManifoldsBase.jl – an interface for manifolds in Julia},
pdfsubject = {JuliaCon 2025 Proceedings},
pdfauthor = {Mateusz Baran, Ronny Bergmann},
pdfkeywords = {Julia, Riemannian manifolds, Lie groups, differential geometry, numerical analysis, scientific computing},
}

% TODO : How to specify an email-address?
\maketitle

\begin{abstract}
    We present an overview about the interface of the Julia package \texttt{ManifoldsBase.jl}
    to define and work with manifolds in Julia.
    The package is a central component of the \texttt{JuliaManifolds} ecosystem and is used by
    nearly all other packages in the ecosystem and several packages in the Julia community.

    We discuss the main design ideas of the interface and present an overview of the features
    and functions provided in \verb|ManifoldsBase.jl| \ManifoldsBaseVersion.
\end{abstract}
%
\section{Introduction}%
\label{sec:Introduction}
%
In many scenarios one encounters data that does not lie in a Euclidean space.
Informally phrased, for certain data, we are not allowed do just add or scale them.
Examples are rotations, data on spheres, in hyperbolic spaces, like when working with general relativity,
symmetric positive definite matrices, like when working with covariance matrices,
or when the data is either bases of subspaces or subspaces themselves, i.e.\ data on the
Stiefel or Grassmann manifold.
The data of interest for this paper still bear enough “structure”, i.e. they live on smooth manifolds.

In a prior paper about \verb|Manifolds.jl|~\cite{AxenBaranBergmannRzecki:2023}, a partial presentation of the interface
of \verb|ManifoldsBase.jl| was given, when the interface was still in an early stage.
A few fundamental design decissions have changed since then.
\\
This paper presents the current state of the interface of \verb|ManifoldsBase.jl| \ManifoldsBaseVersion,
which has reached a stable state.
The paper is organized as follows: in Section~\ref{sec:Notation} we introduce the mathematical background and notation.
In Section~\ref{sec:DesignPrinciples} we present the main design principles and structure of the interface.
Section~\ref{sec:Interface} presents the interface components in detail.
In Section~\ref{sec:Example} we present an example of defining a new manifold,
and Section~\ref{sec:Usage} presents an overview of packages using \verb|ManifoldsBase.jl|.

\section{Mathematical Background}%
\label{sec:Notation}

\begin{figure}
    \centering
    \includegraphics[width=0.15\textwidth]{logo.png}
    \caption*{\ \\[-.5\baselineskip]Logo of \texttt{ManifoldsBase.jl}.}%
    \label{fig:manifoldsbase_logo}
\end{figure}

\section{Components and design principles of the interface}\label{sec:DesignPrinciples}

\subsection{Main types}

\subsection{General function design and a scheme of 3 layers}

\subsection{A trait system to avoid code duplication.}

\section{The interface}\label{sec:Interface}

\subsection{Topological functions}

\subsection{Metric and connection related functions}
\subsection{???}

\section{An example}\label{sec:Example}

\subsection{Defining an own manifold}

\subsection{Adding a second metric to an existin manifold}

\subsection{Implementing a generic algorithm on a manifold}

\section{Where the interface is used}\label{sec:Usage}
\begin{itemize}
    \item Manifolds.jl
    \item Manopt.jl \cite{Bergmann:2022}
    \item ManifoldDiff.jl
    \item ManifoldDiffEq.jl
    \item GeometricKalman.jl
    \item ExponentialFamilyManifolds.jl (Mykola) - and a second one he built on that?
    \item ROMe.jl (?)
\end{itemize}
\label{sec:Example}
\input{bib.tex}
\end{document}